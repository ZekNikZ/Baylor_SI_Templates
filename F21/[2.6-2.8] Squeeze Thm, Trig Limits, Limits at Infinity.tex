\documentclass{siproblemset}

\usepackage{multicol}
\usepackage{xcolor}

% SI Session Information
\course{MTH 1321}
\sessionnum{5}
\sessiondate{9/13/21}

\warmup{Concept Review}
\topic{Trigonometric Limits}
\topic{Limits at Infinity}
\topic{The Squeeze Theorem}
\cooldown{Q\&A}

% Worksheet Information
\title{The Squeeze Thm., Trig.\linebreak Limits, Limits at Infinity}
\sections{Section 2.5b-2.7}
\withnamespace

\begin{document}
    \maketitle
    
    \activity{Warmup}{Concept Review and Overview}{Try these problems \textbf{alone} as your peers join the session. Do your best to not refer to your notes.}{5 minutes}
    
    \frq{What is the Squeeze Theorem?}
    \Tinysp
    
    \mcq[2]{Give the values of the following limits.}{
        \task $\lim\limits_{u\to 0}\dfrac{\sin u}{u}=\lim\limits_{u\to 0}\dfrac{u}{\sin u}=$
        \task $\lim\limits_{u\to 0}\dfrac{1-\cos u}{ u}=$
        \task $\lim\limits_{u\to -\infty}\dfrac{1}{u}=$
        \task $\lim\limits_{u\to \infty}\dfrac{1}{u}=$
        \task $\lim\limits_{u\to -\infty}u^n=~~~~~~~~~~~~(n>0\text{ and even})$
        \task $\lim\limits_{u\to \infty}u^n=~~~~~~~~~~~~(n>0\text{ and even})$
        \task $\lim\limits_{u\to -\infty}u^n=~~~~~~~~~~~~(n>0\text{ and odd})$
        \task $\lim\limits_{u\to \infty}u^n=~~~~~~~~~~~~(n>0\text{ and odd})$
        \task $\lim\limits_{u\to -\infty}b^u=~~~~~~~~~~~~(b>1)$
        \task $\lim\limits_{u\to \infty}b^u=~~~~~~~~~~~~(b>1)$
        \task $\lim\limits_{u\to -\infty}\paren{\dfrac1b}^u=~~~~~~~~~~~~(b>1)$
        \task $\lim\limits_{u\to \infty}\paren{\dfrac1b}^u=~~~~~~~~~~~~(b>1)$
        \task $\lim\limits_{u\to \infty}\dfrac{a_0x^m+a_1x^{m-1}+\cdots}{b_0x^n+b_1x^{n-1}+\cdots}=$   
    }


    \frq{Given $g(x)=\dfrac{1}{x}$, is there some $c$ such that $g(c)=\dfrac34$ where $1\leq c\leq 2$?}
    
    \pagebreak
    
    \activity{Activity 1}{Trig Limits}{Work together with those \textbf{with the same colored worksheet} to answer these questions. Do your best to not refer to your notes while working on these problems.}{15 minutes}
     
    \frq{Evaluate the limit: $\lim\limits_{t\to0}\dfrac{t}{\sin(4t)}$}
    \normalsp
    \frq{Evaluate the limit: $\lim\limits_{\theta\to0}\dfrac{\sin^2(\theta)-2+\cos(2\theta)+\cos^2(\theta)}{\theta}$}
    \Normalsp
    \frq{Evaluate the limit: $\lim\limits_{t\to0}\dfrac{\sin(\pi t)}{3t}$}
    \normalsp
    \pagebreak
    
    \frq{Evaluate the limit: $\lim\limits_{x\to0}x\cot x$}
    \normalsp
    \frq{Evaluate the limit: $\lim\limits_{x\to\pi/2}x\csc x$}
    \normalsp
    \frq{Evaluate the limit: $\lim\limits_{t\to0}\dfrac{\sec(x)-1}{x}$}
    \normalsp

    \pagebreak
    \activity{Activity 2}{Limits at Infinity}{Work together with those \textbf{with different colored worksheet} to answer these questions. Do your best to not refer to your notes while working on these problems.}{15 minutes}
    
    \mcq{Evaluate both $\lim\limits_{x\to-\infty}f(x)$ and $\lim\limits_{x\to\infty}f(x)$ for the following functions.}{
        \task $f(x)=18x^3-4x^7+9$
        \Hugesp
        \task $f(x)=\dfrac{x^3-2x+11}{3-6x^5}$
        \Largesp
    }
    \pagebreak
    
    \frq{Evaluate the limit: $\lim\limits_{t\to\infty}\dfrac{t^{5/2}+3\sqrt{t}}{\sqrt{t}}$}
    \Normalsp
    \frq{Evaluate the limit $\lim\limits_{x\to-\infty}\dfrac{\sqrt{7+9x^2}}{1-2x}$}
    \Normalsp
    \begin{multipartquestion}
        \frq{How many horizontal asymptotes \textit{can} a function have?}
        \tinysp
        \frq{How many vertical asymptotes \textit{can} a function have?}
        \tinysp
    \end{multipartquestion}
    \frq{Find the horizontal asymptotes, if any of $f(x)=\dfrac{10}{1+3^{-x}}$}
    \Smallsp
    
    \activity{Activity 3}{Squeeze Theorem}{Work together with the same groups as last time to answer these questions. Do your best to not refer to your notes while working on these problems.}{15 minutes}
    
    \mcq{Compute the following limits.}{
        \task $\lim\limits_{x\to 2}(x-2)\sin\paren{\dfrac{3}{2-x}}$
        \hugesp
        \task $\lim\limits_{\theta\to 0}\paren{1+\theta^2\cos\paren{e^{-\frac{1}{\theta}}}}$
    }
    
\end{document}