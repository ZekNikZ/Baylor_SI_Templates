\documentclass{siproblemset}

\usepackage{multicol}
\usepackage{tabularx}

\setlength\parindent{0pt}
\linespread{1.5}

% SI Session Information
\course{MTH 1321}
%\sessionnum{1}
\sessiondate{}

% Black answers
%\newcommand{\anscolor}{}

% Red answers
%\newcommand{\anscolor}{\color{red}}

% Blanks
\newcommand{\anscolor}{\color{white}\hspace{0.5cm}}

% Worksheet Information
\title{Reference Sheet}

\begin{document}
    \maketitle
    
    \section{General Equations}
    
    Equation of a line\dots
    
    \dots given the slope and the $y$-intercept: {\Large\anscolor  $y=mx+b$}
    
    \dots given the slope and a point: {\Large\anscolor  $y-y_1=m(x-x_1)$}
    
    \vspace{0.5cm}
    
    Quadratic formula: {\Large\anscolor  $y=\dfrac{-b\pm\sqrt{b^2-4ac}}{2a}$}
    
    \vspace{0.5cm}
    
    Distance formula: {\Large\anscolor  $d=\sqrt{(x_2-x_1)^2+(y_2-y_1)^2}$}
    
    \section{Exponential Identities}
    {\Large \begin{tabularx}{\textwidth}{XX}
    	$(a^b)(a^c)=\anscolor a^{b+c}$   & $\dfrac{a^b}{a^c}=\anscolor a^{b-c}$ \\
    	$(a^b)^c=\anscolor a^{bc}$        & $(ab)^c=\anscolor a^cb^c$            \\
    	$a^{b/c}=\anscolor \sqrt[c]{a^b}$  &                                       \\
    	$a^0=\anscolor 1$                  & $a^1=\anscolor a$
    \end{tabularx}}

    \section{Logarithmic Identities}
    Definition of the logarithm: {\Large\anscolor  $a^b=c ~~ \equiv ~~ \log_a(c)=b$}
    
    {\Large \begin{tabularx}{\textwidth}{XX}
    	$\log(a^b)=\anscolor b\log(a)$ & $\log(ab)=\anscolor \log(a)+\log(b)$  \\
    	$a^{\log_a(b)}=\anscolor b$    & $\log(a/b)=\anscolor \log(a)-\log(b)$ \\
    	$\ln(e)=\anscolor 1$           & $\ln(1)=\anscolor 0$
    \end{tabularx}}
    \newpage
    
    \section{Trigonometric Identities}
    {\Large \begin{tabularx}{\textwidth}{XX}
    	$\sin x=\dfrac{\anscolor O}{\anscolor H}=\dfrac{\anscolor 1}{\anscolor \csc{x}}$\vspace{0.25cm}                                               & $\csc x=\dfrac{\anscolor H}{\anscolor O}=\dfrac{\anscolor 1}{\anscolor \sin{x}}$\vspace{0.25cm}                                               \\
    	$\cos x=\dfrac{\anscolor A}{\anscolor H}=\dfrac{\anscolor 1}{\anscolor \sec{x}}$\vspace{0.25cm}                                               & $\sec x=\dfrac{\anscolor H}{\anscolor A}=\dfrac{\anscolor 1}{\anscolor \cos{x}}$\vspace{0.25cm}                                               \\
    	$\tan x=\dfrac{\anscolor O}{\anscolor A}=\dfrac{\anscolor 1}{\anscolor \cot{x}}=\dfrac{\anscolor \sin x}{\anscolor \cos{x}}$\vspace{0.25cm} & $\cot x=\dfrac{\anscolor A}{\anscolor O}=\dfrac{\anscolor 1}{\anscolor \tan{x}}=\dfrac{\anscolor \cos x}{\anscolor \sin{x}}$\vspace{0.25cm}
    \end{tabularx}}


    {\Large \begin{tabularx}{\textwidth}{X}
    	$\paren{{\anscolor \sin x}}^2+\paren{{\anscolor \cos x}}^2=1$ \\ $\paren{{\anscolor \tan x}}^2+1=\paren{{\anscolor \sec x}}^2$ \\ $\paren{{\anscolor \cot x}}^2+1=\paren{{\anscolor \csc x}}^2$
    \end{tabularx}}
    
\end{document}