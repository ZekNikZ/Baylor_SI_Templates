\documentclass{siproblemset}

% SI Session Information
\course{MTH 1321}       % the course of your SI
\sessionnum{17}         % (optional) specify the session number
\sessiondate{11/1/21}   % the date of the session

\warmup{Concept Review}
\topic{Curve Sketching}
%\topic{Trigonometric Optimization}
%\cooldown{Finding the Objective Function}

% Worksheet Information
\title{Applied Optimization}
\sections{Section 4.6}
\withnamespace

\begin{document}
    \maketitle
    
    \activity{Warmup}{Concept Review}{Try these problems \textbf{alone} as your peers join the session. Do your best to not refer to your notes.}{15 minutes}
    
%    \frq{What are the steps to optimize a function?}
%    \Largesp
    \frq{What are some things that we can use to help us sketch a function?}
    \newpage
    
    \activity{Activity 1}{Curve Sketching}{Work together in your \textbf{small groups} to answer these questions. Do your best to not refer to your notes while working on these problems.}{30 minutes}
    \frq{Sketch the curve $f(x)=x^3-6x^2+9x+2$}
    \newpage
    
    \frq{Sketch the curve $g(x)=\dfrac{5x^2}{x^2+9}$}
    \newpage
    
%    \activity{Activity 2}{Optimization}{Work together in your \textbf{breakout rooms} to answer these questions. Do your best to not refer to your notes while working on these problems.}{30 minutes}
%    
%    \mcq{Your task is to design a rectangular industrial warehouse consisting of three separate spaces of equal size as in the figure below. The wall materials cost \$500 per linear meter and your company allocates \$2,400,000 for that part of the project involving the walls.}{
%        \task Which dimensions maximize the area of the warehouse?
%        \task What is the area of each compartment in this case?
%    }
%    \begin{center}
%        \begin{tikzpicture}
%            \filldraw[fill=lightgray] (0,0) rectangle (6,3);
%            \draw[black, ultra thick, line cap=rect] (0,0) -- (6,0);
%            \node at (3,0) [below,align=center]{Top-down view of warehouse};
%            \draw[black, ultra thick, line cap=rect] (0,3) -- (6,3);
%            \draw[black, ultra thick] (0,0) -- (0,3);
%            \draw[black, ultra thick] (2,0) -- (2,3);
%            \draw[black, ultra thick] (4,0) -- (4,3);
%            \draw[black, ultra thick] (6,0) -- (6,3);
%        \end{tikzpicture}
%    \end{center}
%    \newpage
%
%    \activity{Bonus problem}{Trigonometric Optimization}{Work together in your \textbf{breakout rooms} to answer these questions. Do your best to not refer to your notes while working on these problems.}{30 minutes}
%    
%    \frq{The force $F$ (in Newtons) required to move a box of mass $m$ kg in motion by pulling on an attached rope (see the figure below) is $$F(\theta)=\frac{\mu_k m g}{\cos\theta+\mu_k\sin\theta}$$ where $\theta$ is the angle between the rope and the horizontal, $\mu_k$ is the coefficient of kinetic friction and $g=\v{9.81}$ is the acceleration due to gravity. Find the angle $\theta$ that minimizes the required force $F$, assuming that $\mu_k=0.5$.}
%    \begin{center}
%        \begin{tikzpicture}
%            \fill[fill=lightgray] (-1,0) rectangle (6,-0.5);
%            \draw[ultra thick] (-1,0) -- (6,0);
%            \draw[thick] (0,0) rectangle (3,2) node[midway]{$m$};
%            \draw[->,ultra thick] (3,1) -- (6,3) node[above]{$\vec F$};
%            \draw[dashed] (3,1) -- (6,1);
%            \draw (4,1) arc (0:34:1);
%            \node at (4,1.6) [below right]{$\theta$};
%        \end{tikzpicture}
%    \end{center}
%    \newpage
%    
%    \activity{Cooldown}{Finding the Objective Function}{Work \textbf{as a class} to answer these questions. Try not to use your notes.}{15 minutes}
%    
%    \mcq{For each scenario below, find the objective/key function.}{
%        \task Find two positive numbers such that their product is 192 and the sum of the first plus three times the second is as small as possible.
%        \Smallsp
%        \task Find the dimensions of the rectangle of maximum area that can be inscribed in a circle of radius $\SI{4}{\meter}$.
%        \Smallsp
%        \task A \SI{5}{\meter} piece of string is bent into an L-shape. Where should the bend be made to minimize the distance between the two ends?
%        \Smallsp
%        \task A rancher will use 600 feet of fencing to build a corral in the shape of a semicircle on top of a rectangle. Find the dimensions that maximize the area of the corral.
%    }
\end{document}