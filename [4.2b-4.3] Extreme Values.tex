\documentclass{siproblemset}

% SI Session Information
\course{MTH 1321}       % the course of your SI
\sessionnum{16}         % (optional) specify the session number
\sessiondate{10/26/20}   % the date of the session

\warmup{Concept Review}
\topic{Extreme Values}
\cooldown{Rolle's Theorem}

% Worksheet Information
\title{Extreme Values}
\sections{Sections 4.2b}
\withnamespace

\begin{document}
    \maketitle
    
    \activity{Warmup}{Concept Review}{Do these problems \textbf{alone} then discuss your answers with a partner. Try not to use your notes.}{15 minutes}

    \frq{What are critical values?}
    \Tinysp
    \frq{What is the Extreme Value Theorem?}
    \Smallsp
    \frq{How do we find absolute maxes and mins?}
    \Smallsp
    \frq{What are the critical value(s) of the function $f(x)=x^2+\dfrac{16}{x}$?}
    \newpage

    \activity{Activity 1}{Extreme Values}{Make a \textbf{group of two or three, all with different colored worksheets,} to answer your assigned question. Then, try to answer the other questions. Try not to use your notes.}{30 minutes}
    
    \mcq{Find the absolute maximum and minimum of the functions on the given intervals, if possible:}{
        \task $f(x)=x^3-12x^2+21x-7~~~~~[-2,2]$
        \Normalsp
        \task $g(x)=x-\dfrac{4x}{x+1}~~~~~[0,3]$
        \Normalsp
        \task $h(x)=\sqrt{2x+x^2}~~~~~[0,2]$
        \Normalsp
        \task $j(x)=x^3-3x+1~~~~~[-3,3]$
        \Normalsp
    }


    \activity{Activity 2}{Extreme Values, continued}{Make a \textbf{group of two or three, all with different colored worksheets,} to answer your assigned question. Then, try to answer the other questions. Try not to use your notes.}{30 minutes}
    \mcq{Find the absolute maximum and minimum of the functions on the given intervals, if possible:}{
        \task $f(x)=x^2e^{-x}~~~~~[-2,3]$
        \Normalsp
        \task $g(x)=x-\dfrac{4x}{x+1}~~~~~[-2,2]$
        \Normalsp
        \task $h(x)=\sqrt{\paren{x+3}^3}~~~~~[1,5]$
        \largesp
        \task $j(x)=\ln(x+1)~~~~~[-2,4]$
        \Normalsp
        \task $k(x)=\ln(x+1)~~~~~[2,4]$
        \Normalsp
    }
    \newpage
    
    \activity{Cooldown}{Extreme Values From a Graph}{Do this problems \textbf{alone}.}{15 minutes}
    
    \frq{What is Rolle's Theorem?}
    \Smallsp
    
    \frq{Demonstrate that the equation there exists a value $-3\leq c\leq 3$ such that $f'(c)=0$ for the function $f(x)=6-x^2$}
    
\end{document}