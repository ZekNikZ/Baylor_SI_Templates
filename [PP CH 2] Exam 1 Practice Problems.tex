\documentclass{siproblemset}

\usepackage{multicol}
\usepackage{xcolor}
\usepackage{mathtools}
\usepackage{enumitem}

% SI Session Information
\course{MTH 1321}
\sessionnum{PP1}
\sessiondate{2/8/21}

% Worksheet Information
\title{Exam 1 Practice Problems}
\withnamespace

\definecolor{darkred}{RGB}{110,0,0}
\raggedcolumns

%\debugmode

\begin{document}
    \maketitle
    
    \begin{center}
        \framebox{
            \begin{minipage}{\textwidth}
                \begin{center}
                    \textbf{This list of problems is not intended to be comprehensive of\linebreak all the material on your exam, but it should be good practice.}
                \end{center}
                \begin{enumerate}[leftmargin=*]
                    \item This list consists of homework questions which I consider to be representative of the important topics of each section.
                    \item Work through some or all of these leading up to your exam.
                    \item This list is intended to be \underline{supplemental} to your own exam review. Please do not only use this and the practice test to study for your exam.
                \end{enumerate}
                \begin{center}
                    \color{darkred}\textbf{ Please do not share this problem list with anyone else. \underline{Your} commitment to SI and reviewing material earned this, not anyone else's.}
                \end{center}
            \end{minipage}
        }
    \end{center}

    \Large
%    \begin{multicols*}{1}
        \underline{\textbf{Chapter 2: Limits}}
        
        \textbf{2.1}: 2, 11
        
        \textbf{2.2}: 21, 26, 47
        
        \textbf{2.3}: 17, 31, 32
        
        \textbf{2.4}: 10, 31, 53, 79, 82, 86
        
        \textbf{2.5}: 1, 15, 27
        
        \textbf{2.6}: 2, 25, 33
        
        \textbf{2.7}: 8, 10, 15, 39, 42
        
        \textbf{2.8}: 9, 20
        
%        \ \linebreak
%        
%        \underline{\textbf{Chapter 3: Derivatives}}
%        
%        \textbf{3.1}: 4, 8, 17, 32
%        
%        \textbf{3.2}: 19, 34, 55
%        
%        \textbf{3.3}: 23, 36, 45, 46, 47, 48
%        
%        \textbf{3.4}: 2, 6, 41
%        
%        \textbf{3.5}: 15, 24
%        
%        \textbf{3.6}: 5, 20, 30
%        
%        \textbf{3.7}: 14, 38, 48, 66
%        
%        \textbf{3.8}: 17, 24, 38 39, 46
%        
%        \textbf{3.9}: 13, 18, 33, 48
%        
%        \textbf{3.10}: 17, 21, 25, 38
%        
%        \columnbreak
%        
%        \underline{\textbf{Chapter 4: Derivative Applications}}
%        
%        \textbf{4.1}: 4, 21, 23, 36
%        
%        \textbf{4.2}: 1, 5, 21, 52, 90, 91
%        
%        \textbf{4.3}: 4, 9, 15, 16, 17, 20, 33, 45, 56
%        
%        \textbf{4.4}: 2, 7, 13, 36, 41, 56, 65
%        
%        \textbf{4.6}: 9, 22, 31, 54
%        
%        \textbf{4.7}: 8, 15, 24, 29, 47, 74
%        
%        \ \linebreak
%        
%        \underline{\textbf{Chapter 5: Integrals}}
%        
%        \textbf{5.1}: 6, 15, 18, 23, 30, 50
%        
%        \textbf{5.2}: 9, 14, 15, 16, 41, 46, 48, 57, 58, 59, 60
%        
%        \textbf{5.3}: 15, 24, 27, 33, 53, 66
%        
%        \textbf{5.4}: 9, 14, 24, 28, 36, 40, 48
%        
%        \textbf{5.5}: 2, 12, 16, 22, 28, 35, 47
%        
%        \textbf{5.6}: 1, 7, 10, 15, 22
%        
%        \textbf{5.7}: 8, 19, 24, 42, 45, 65, 90, 91, 93
%        
%        \textbf{5.8}: 4, 10, 20, 23, 28, 29, 61, 68
%    \end{multicols*}
    
\end{document}