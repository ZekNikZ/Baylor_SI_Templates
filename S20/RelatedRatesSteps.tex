\documentclass{siproblemset}

% Worksheet Information
\title{Related Rates Steps}
%\sections{Section 3.10}
%\withnamespace

\raggedright

\begin{document}
    \maketitle
    
    \begin{enumerate}
        \item \textbf{Draw a picture of the problem and label everything relevant (if not given one).}
        
        \textit{Example: Draw a triangle and label the sides $x$, $y$, and $z$.}
        \item \textbf{Collect givens and identify the ``moment'' or ``when''.}
        
        \textit{Hint: Create a box of all variables and their derivatives and fill in as much as you can. The ``moment'' is going to be a specified time or measurement.}
        \item \textbf{Determine your ``end goal'' (i.e. the thing you are solving for, usually a derivative).}
        
        \textit{Hint: Figure out which of the unknown derivatives you need to solve for and mark it in some way so you can keep track of it.}
        \item \textbf{Collect all of the equations that you can think of which relate the quantities.}
        
        \textit{Hint: Create a box of relevant equations, like similar triangles, areas, the Pythagorean Theorem, etc.}
        \item \textbf{Choose the ``main'' equation that you will be working with from the list that you generated.}
        
        \textit{Example: If the problems wants you to find the change in height of a sliding ladder, the Pythagorean Theorem might be your ``main'' equation.}
        \item \textbf{Rewrite the ``main'' equation in terms of as few variables as possible (if applicable).}
        
        \textit{Hint: Use the other equations to rewrite the ``main'' equation in just two or three variables, if possible. However, if the problem gave you the value of more than one derivative, you can usually skip this step.}
        
        \textit{Example: In a volume problem, you often have to write volume in terms of just radius or just height.}
        \item \textbf{Differentiate the ``main'' equation.}
        
        \textit{Hint: You will most likely have to use implicit differentiation at this step.}
        \item \textbf{Plug in any given values.}
        
        \textit{Hint: If you still don't have the value of a variable, use the other equations to find it.}
        
        \textit{Common Mistake: Make sure to do this step \underline{after} you differentiate.}
        \item \textbf{Solve for the derivative or quantity you want (i.e. the ``end goal'').}
        
        \textit{Hint: Use the same process as when we solved for $\dddx[y]$ in implicit differentiation problems.}
        \item \textbf{Write your answer in terms of the problem including units.}
        
        \textit{Hint: If you can't figure out the units, look at the derivative: $\dddf[V]{t}\Rightarrow\dfrac{\text{units of volume}}{\text{units of time}}$.}
        
        \textit{Common Mistake: If your answer for a derivative is \underline{negative}, it means that that quantity is \underline{decreasing}. So, when you write your answer, keep that in mind.}
        
        \textit{Example: If your answer for a derivative is $\dddf[V]{t}=-2$, you may write ``volume is \underline{decreasing} at a rate of $\SI{2}{\meter^3/\second}$''.}
    \end{enumerate}
    
\end{document}