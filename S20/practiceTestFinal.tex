\documentclass{siproblemset}

\usepackage{multicol}
\usepackage{xcolor}
\usepackage{mathtools}

% SI Session Information
\course{MTH 1321}
\sessionnum{FPT}
\sessiondate{5/3/21}

% Worksheet Information
\title{Final Exam Practice Test}
\withnamespace

\definecolor{darkred}{RGB}{110,0,0}

%\debugmode

\begin{document}
    \maketitle
    
    \begin{center}
        \framebox{
            \begin{minipage}{\textwidth}
                \begin{center}
                    \textbf{When completing this practice test, do your best to mimic the test environment:}
                \end{center}
                \begin{enumerate}
                    \item Do not use a calculator.
                    \item Try not to use your notes.
                    \item Time yourself, make sure you are completing the problems at a comfortable pace. Remember that you will only get 2 hours for the actual exam (with fewer questions of course).
                    \item Do trivial simplifications in your answers.
                \end{enumerate}
                \begin{center}
                    \color{darkred}\textbf{ Please do not share this practice test with anyone else. \underline{Your} commitment to SI and reviewing material earned this, not anyone else's.}
                \end{center}
                {\centering When you have finished the practice test, check Canvas to check your answers.\\}
            \end{minipage}
        }
    \end{center}

    % Chapter 2
    \frq{For $f(x)=\sqrt{2x}$ compute the slopes of the secant lines from 16 to each of $16\pm 0.01$, $16\pm 0.001$, $16\pm 0.0001$ and use those values to estimate the slope of the tangent line at $x=16$. \textit{Note that this problem requires a calculator. You will not get a calculator on your final, but this is good practice nonetheless.}}
    \newpage
    \mcq{Evaluate the limit if it exists. If not, indicate whether it approaches $\infty$ or $-\infty$.}{
        \task $\lim\limits_{x\to-1}\dfrac{3x^2+4x+1}{x+1}$
        \tinysp
        \task $\lim\limits_{x\to-2}\dfrac{5}{x^4}$
        \tinysp
        \task $\lim\limits_{x\to1}\dfrac{x^3-x}{x-1}$
        \smallsp
        \task $\lim\limits_{x\to0}\dfrac{e^{3x}-e^x}{e^x-1}$
        \Smallsp
        \task $\lim\limits_{\theta\to0}\dfrac{\sin5\theta}{\theta}$
        \Smallsp
        \task $\lim\limits_{x\to0}\dfrac{\sin4x}{\sin3x}$
        \Smallsp
    }
    \frq{Find a constant $b$ such that $h$ is continuous at $x=2$, where}
    $$h(x)=\begin{cases} 
    x+1 & \text{for }\abs{x}<2 \\
    b-x^2 & \text{for }\abs{x}\geq2
    \end{cases}$$
    \Smallsp
    \frq{Find the horizontal asymptote(s) of the following function:}
    $$f(x)=\frac{9x^2-4}{2x^2-x}$$
    \Smallsp
    \mcq[2]{Calculate (a)-(d), assuming that $$\lim\limits_{x\to3}f(x)=6~~~~~\lim\limits_{x\to3}g(x)=4$$}{
        \task $\lim\limits_{x\to3}(f(x)-2g(x))$
        \smallsp
        \task $\lim\limits_{x\to3}x^2f(x)$
        \smallsp
        \task $\lim\limits_{x\to3}\dfrac{f(x)}{g(x)+x}$
        \task $\lim\limits_{x\to3}(2g(x)^3-g(x)^{3/2})$
    }
    \newpage
    \mcq[2]{Sketch the graph of a function $f$ such that }{
        \task $\lim\limits_{x\to2^-}f(x)=1$
        \task $\lim\limits_{x\to2^+}f(x)=3$
        \task $\lim\limits_{x\to4}f(x)$ exists but does not equal $f(4)$
    }
    \Smallsp
    \mcq{True or false?}{
        \task If $\lim\limits_{x\to3}f(x)$ exists, then $\lim\limits_{x\to3}f(x)=f(3)$.
        \task If $\lim\limits_{x\to0}\dfrac{f(x)}{x}=1$, then $f(0)=0$.
        \task If $\lim\limits_{x\to-7}f(x)=8$, then $\lim\limits_{x\to-7}\dfrac{1}{f(x)}=\dfrac18$.
        \task If $\lim\limits_{x\to5^+}f(x)=4$ and $\lim\limits_{x\to5^-}=8$, then $\lim\limits_{x\to5}f(x)=6$.
        \task If $\lim\limits_{x\to0}\dfrac{f(x)}{x}=1$, then $\lim\limits_{x\to0}f(x)=0$.
        \task If $\lim\limits_{x\to5}f(x)=2$, then $\lim\limits_{x\to5}f(x)^3=8$
    }
    \frq{Use the Intermediate Value Theorem to prove that the curves $y=x^2$ and $y=\cos x$ intersect.}
    \newpage
    
    % Chapter 3
    \frq{Compute $f'(a)$ using the limit definition of the derivative and find an equation of the tangent line to the graph of $f$ at $x=a$.$$f(x)=\dfrac{1}{x-3}$$}
    \Largesp
    \frq{Each graph in the figure below shows the graph of a function $f$ and its derivative $f'$. Determine which is the function and which is the derivative.}
    \begin{center}
        \includegraphics[width=0.6\linewidth]{"FigureCH3-2"}
    \end{center}
    
    \frq{Use the table below of the number $A(t)$ of automobiles (in millions) manufactured in the United States in year $t$ to complete this problem. What is the interpretation of $A'(t)$? Estimate $A'(1971)$. Does $A'(1974)$ appear to be positive or negative?}
    \begin{center}
        \begin{tabular}{|l|c|c|c|c|c|c|c|}
            \hline
            t&1970&1971&1972&1973&1974&1975&1976\\
            \hline
            $A(t)$&6.55&8.58&8.83&9.67&7.32&6.72&8.50\\
            \hline
        \end{tabular}
    \end{center}
    \newpage
    \mcq{Compute the derivative.}{
        \task $y=3x^5-7x^2+4$
        \Tinysp
        \task $y=\sqrt{x+\sqrt{x+\sqrt{x}}}$
        \Tinysp
        \task $y=\dfrac{1}{1+\sec t}$
        \Tinysp
        \task $y=z\csc(9z+1)$
        \Tinysp
        \task $f(x)=\ln(4x^2+1)$
        \Tinysp
        \task $f(x)=7^{-2x}$
        \Tinysp
        \task $f(x)=(\cos^2x)^{\cos x}$
        \Tinysp
    }
\newpage
    \frq{Find the $x$-values of the points on the graph of $f(x)=x^3-3x^2+x+4$ where the tangent line has slope 10.}
    \Tinysp
    \frq{Compute $\dfrac{\text{d}y}{\text{d}x}$ and $\dfrac{\text{d}^2y}{\text{d}x^2}$ given $6xy+y^2=10$.}
    \Normalsp
    \frq{Consider a cylinder whose volume is increasing at a rate of $2 \text m^2/\text s$  and whose radius is decreasing at a rate of $1 \text m/\text s$. What is the rate of change of the height of the cylinder when its diameter is $2 \text m$ and its volume is $10 \text m$?}
    \newpage
    \frq{A light moving at $\SI{0.8}{\meter/\second}$ approaches a man standing $\SI{4}{\meter}$ from a wall (as seen below). The light is $\SI{1}{\meter}$ above the ground. How fast is the tip $P$ of the man's shadow moving when the light is $\SI{7}{\meter}$ from the wall?}
    \begin{center}
        \includegraphics[width=0.6\linewidth]{"FigureCH3-11"}
    \end{center}
    \Largesp
    
    % Chapter 4
    \frq{Find the linearization of $v(t)=32t-4t^2$ at $t=2$.}
    \newpage
    \frq{Verify the Mean Value Theorem for $f(x)=\ln x$ on $[1,4]$.}
    \normalsp
    \frq{Find the critical points and determine whether they are minima, maxima, or neither:}
    $$f(x)=x^{2/3}(1-x)$$
    \Normalsp
    \frq{A function $f$ has derivative $f'(x)=\dfrac{1}{x^4+1}$. Where on the interval $[1,4]$ does $f$ take on its maximum value?}
    \newpage
    \frq{Find the extreme values of $R(t)=\dfrac{t}{t^2+t+1}$ on the interval $[0,3]$.}
    \Normalsp
    \frq{Find the points of inflection of $f(x)=(x^2-x)e^{-x}$.}
    \Normalsp
    \frq{Draw a curve $y=f(x)$ for which $f'$ and $f''$ have signs as indicated below:}
    \begin{center}
        \includegraphics[width=0.6\linewidth]{"FigureCH4-2"}
    \end{center}
    \newpage
    \frq{Sketch the graph of $y=12x-3x^2$, including the transition points and asymptotic behavior.}
    \newpage
%    \frq{A rectangular open-topped box of height $h$ with a square base of side $b$ has volume $V=\SI{4}{\meter^3}$. Two of the side faces are made of material costing $\$40/\si{\meter^2}$. The remaining sides cost $\$20/\si{\meter^2}$. Which values of $b$ and $h$ minimize the cost of the box?}
%    \Hugesp
%    \frq{Find the dimensions of a cylindrical can with a bottom but no top of volume $\SI{4}{\meter^3}$ that uses the least amount of metal.}
%    \Hugesp
    
    % Chapter 5
    \frq{Estimate $R_4$, $L_4$, and $M_4$ on $[1,3]$ for the function shown below:}
    \begin{center}
        \includegraphics[width=0.4\linewidth]{"FigureCH5-1"}
    \end{center}
    \Hugesp
    \mcq{Compute the integral.}{
        \task $\int(4x^3-2x^2)\dd x$
        \Smallsp
        \task $\int\tan3\theta\sec3\theta\dd\theta$
        \Smallsp
        \task $\int\dfrac{(x^2+1)\dd x}{(x^3+3x)^4}$
        \Smallsp
        \task $\int\dfrac{x^5+3x^4}{x^2}\dd x$
        \Smallsp
        \task $\int_{-3}^{3}\abs{x^2-4}\dd x$
        \Smallsp
        \task $\int\dfrac{\dd t}{t(1+(\ln t)^2)}$
        \Smallsp
        \task $\int\dfrac{\dd x}{x\sqrt{\ln x}}$
        \Smallsp
        \task $\int\dfrac{\dd x}{\sqrt{9-4x^2}}$
        \Smallsp
    }
    \frq{Solve the differential equation $\dddx[y]=4x^3$ given that $y(1)=4$.}
    \newpage
    \frq{Combine the following to write as a single integral.}
    $$\int_{0}^{8}f(x)\dd x+\int_{-2}^{0}f(x)\dd x+\int_{8}^{6}f(x)\dd x$$
    \smallsp
    \mcq{A particle starts at the origin at time $t=0$ and moves with velocity $v(t)$ as shown below.}{
        \task How many times does the particle return to the origin in the first 12 seconds?
        \task What is the particle's maximum distance from the origin?
        \task What is the particle's maximum distance to the left of the origin?
    }
    \begin{center}
        \includegraphics[width=0.6\linewidth]{"FigureCH5-5"}
    \end{center}
    \newpage
    \mcq{Compute the derivative.}{
        \task $\dddf{y}\int_{-2}^{y}3^x\dd x$
        \smallsp
        \task $G'(2)$, where $G(x)=\int_{0}^{x^3}\sqrt{t+1}\dd t$
        \Smallsp
    }
    
    
    \begin{multipartquestion}
        The function $f$ is continuous for all values of $x$. Information about the sign of $f'$ and $f''$ is organized in the table below.
        
        \begin{center}
            \begin{tabular}{|c|c|c|c|c|}
                \hline
                &$x<1$&$1<x<2$&$2<x<3$&$x>3$\\
                \hline
                Sign of $f'$&$-$&$-$&$+$&$-$\\
                \hline
                Sign of $f''$&$+$&$-$&$-$&$+$\\
                \hline
            \end{tabular}
        \end{center}
        
        Mark each of the following statements as true (T) or false (F). You do \textit{not} need to justify your answer.
        \frq{$f$ has a local minimum at $x=2$}
        \frq{$f$ is decreasing and concave down at $x=4$}
        \frq{$f$ has an inflection point at $x=1$}
        \frq{$f'$ is decreasing at $x=2.5$}
        \frq{$f'$ has a local extremum at $x=1$}
    \end{multipartquestion}
\newpage
    % Misc
    \begin{multipartquestion}
        Researchers are studying the effectiveness of a new antibiotic. Let $P(t)$ be the number of bacteria present $t$ hours after the antibiotic is administered. Data collected by the researchers indicated that the over the first $12$ hours, the number of bacteria decreased while its rate of change increased, and then afterwards the number of bacteria continued to decrease, but its rate of change decreased.
        \mcq{Which first derivative statements best describe this scenario?}{
            \task $P'(t) > 0$ for $0 < t < 12$ and $P'(t) > 0$ for $t > 12$
            \task $P'(t) > 0$ for $0 < t < 12$ and $P'(t) < 0$ for $t > 12$
            \task $P'(t) < 0$ for $0 < t < 12$ and $P'(t) > 0$ for $t > 12$
            \task $P'(t) < 0$ for $0 < t < 12$ and $P'(t) < 0$ for $t > 12$
            \task None of the above
        }
        \mcq{Which second derivative statements best describe this scenario?}{
            \task $P''(t) > 0$ for $0 < t < 12$ and $P''(t) > 0$ for $t > 12$
            \task $P''(t) > 0$ for $0 < t < 12$ and $P''(t) < 0$ for $t > 12$
            \task $P''(t) < 0$ for $0 < t < 12$ and $P''(t) > 0$ for $t > 12$
            \task $P''(t) < 0$ for $0 < t < 12$ and $P''(t) < 0$ for $t > 12$
            \task None of the above
        }
        \mcq{At $t=12$, the graph of $y=P(t)$ has}{
            \task a local maximum
            \task a local minimum
            \task a critical number but no local extremum
            \task an inflection point
            \task None of the above
        }
        \frq{Assuming that there are a positive number of bacteria, $P_0$, present at time $t = 0$ and that at time $t = 24$, all bacteria are eliminated, sketch a possible graph of $P$ corresponding to all the criteria described above.}
    \end{multipartquestion}
    
\end{document}