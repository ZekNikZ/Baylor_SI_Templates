\documentclass{siproblemset}

\usepackage{multicol}
\usepackage{xcolor}
\usepackage{mathtools}

% SI Session Information
\course{MTH 1321}
\sessionnum{PT3}
\sessiondate{4/4/22}

% Worksheet Information
\title{Practice Test \#3}
\sections{Chapter 4}
\withnamespace

\definecolor{darkred}{RGB}{110,0,0}

%\debugmode

\begin{document}
    \maketitle
    
    \begin{center}
        \framebox{
            \begin{minipage}{\textwidth}
                \begin{center}
                    \textbf{When completing this practice test, do your best to mimic the test environment:}
                \end{center}
                \begin{enumerate}
                    \item Do not use a calculator.
                    \item Try not to use your notes.
                    \item Time yourself, make sure you are completing the problems at a comfortable pace. Remember that you will only get \textbf{2 hours} for the actual exam (with fewer questions of course).
                    \item The answer key will be available at the test review session on \textbf{4/4/22 from 5-7 pm in SidRich 224}.
                    \item If you have any questions, feel free to email me at {\color{blue} Matthew\_McCaskill1@baylor.edu}.
                \end{enumerate}
            \end{minipage}
        }
    \end{center}

    \begin{center}
        \framebox{
            \begin{minipage}{\textwidth}
                \begin{center}
                    \textbf{Problem Updates}
                    
                    Updated parts of problems are marked in \textbf{\color{red} RED}.
                \end{center}
                \begin{itemize}
                    \item Problem 1: changed function to simplify non-calculator arithmetic (4/4/22)
                    \item Problem 4: changed interval from $[1,3]$ to $[1,4]$ (4/4/22)
                    \item Problem 12: added note to \textit{not} simplify (4/4/22)
                \end{itemize}
            \end{minipage}
        }
    \end{center}
    \newpage

    \activity{Section 1}{Exam Preparation}{These problems are designed to look closely like what you will encounter on the exam.}{try these problems first}
    
    \begin{multipartquestion}
        \frq{Find the linearization of {\color{red} $f(x)=x^2\ln(x)$} at {\color{red} $x=e$} \textbf{in slope-intercept form}.}
        \Smallsp
        \frq{Use your linearization equation from part (a) to approximate {\color{red} $f(e+1)$}. {\color{red} Leave your answer in terms of $e$.}}
        \Smallsp
    \end{multipartquestion}

    \frq{Find the \textbf{coordinates} of the absolute maximum(s) and minimum(s) of the function below on the given interval.}
    $$f(x)=\sqrt{3}\sin(x)+\cos(x)~~~[0,2\pi]$$
    \newpage

    \begin{multipartquestion}
        \frq{Can we apply Rolle's Theorem to $f(x)=-x^2+6x-6$ on the interval $[1,4]$? Why or why not?}
        \Smallsp
        \frq{Find all numbers $c$ which satisfy the conclusions of Rolle's Theorem on the given interval.}
        \Smallsp
    \end{multipartquestion}

    \begin{multipartquestion}
        \frq{Can we apply the Mean Value Theorem to $f(x)=x^2-2x-8$ on the interval $[-1,{\color{red} 4}]$? Why or why not?}
        \Smallsp
        \frq{Find all numbers $c$ which satisfy the conclusions of the Mean Value Theorem on the given interval.}
        \Smallsp
    \end{multipartquestion}
    \newpage
    
    \begin{multipartquestion}{The graph below is the graph of a function $f$.}
        \begin{multicols}{2}
            \begin{center}
                \includegraphics[width=\linewidth]{img/pt3-graph1}
            \end{center}
            \vfill\null
            \columnbreak
            \frq{On what interval(s) is $f'$ positive?}
            \ \newline
            \frq{On what interval(s) is $f''$ negative?}
            \ \newline
            \frq{At which $x$-value(s) does $f$ have an inflection point?}
            \ \newline
        \end{multicols}
    \end{multipartquestion}

    \vspace{-1cm}
    \begin{multipartquestion}{The graph below is the graph of a derivative function $f'$.}
        \begin{multicols}{2}
            \begin{center}
                \includegraphics[width=\linewidth]{img/pt3-graph1}
            \end{center}
            \vfill\null
            \columnbreak
            \frq{On what interval(s) is $f'$ decreasing?}
            \ \newline
            \frq{What are the critical values of $f$?}
            \ \newline
            \frq{At which $x$-value(s) does $f''$ have a max/min?}
            \ \newline
        \end{multicols}
    \end{multipartquestion}

    \vspace{-1cm}
    \begin{multipartquestion}{The graph below is the graph of a second derivative function $f''$.}
        \begin{multicols}{2}
            \begin{center}
                \includegraphics[width=\linewidth]{img/pt3-graph1}
            \end{center}
            \vfill\null
            \columnbreak
            \frq{On what interval(s) is $f'$ increasing?}
            \ \newline
            \frq{What are the hypercritical values of $f$?}
            \ \newline
            \frq{At which $x$-value(s) does $f'$ have a max/min?}
            \ \newline
        \end{multicols}
    \end{multipartquestion}
    \newpage
    
    \begin{multipartquestion}{The graph below is the graph of a function $f$.}
        \begin{multicols}{2}
            \begin{center}
                \includegraphics[width=\linewidth]{img/pt3-graph2}
            \end{center}
            \vfill\null
            \columnbreak
            \frq{What are the critical values of $f$?}
            \ \newline
            \frq{On what interval(s) is $f'$ decreasing?}
            \ \newline
            \frq{At which $x$-value(s) does $f'$ have a max/min?}
            \ \newline
        \end{multicols}
    \end{multipartquestion}
    
    \vspace{-1cm}
    \begin{multipartquestion}{The graph below is the graph of a derivative function $f'$.}
        \begin{multicols}{2}
            \begin{center}
                \includegraphics[width=\linewidth]{img/pt3-graph2}
            \end{center}
            \vfill\null
            \columnbreak
            \frq{On what interval(s) is $f$ decreasing?}
            \ \newline
            \frq{On what interval(s) is $f$ concave up?}
            \ \newline
            \frq{At which $x$-value(s) does $f$ have a max/min?}
            \ \newline
        \end{multicols}
    \end{multipartquestion}
    
    \vspace{-1cm}
    \begin{multipartquestion}{The graph below is the graph of a second derivative function $f''$.}
        \begin{multicols}{2}
            \begin{center}
                \includegraphics[width=\linewidth]{img/pt3-graph2}
            \end{center}
            \vfill\null
            \columnbreak
            \frq{On what interval(s) is $f$ concave down?}
            \ \newline
            \frq{On what interval(s) is $f'$ concave up?}
            \ \newline
            \frq{At which $x$-value(s) does $f$ have an inflection point?}
            \ \newline
        \end{multicols}
    \end{multipartquestion}
    \newpage
    
    \frq{Jill wants to build a rectangular garden along the side of her house with a fence around it. If she wants the area of the garden to be 200ft$^2$, what is the minimum amount of fencing that she needs to enclose the area? Note: there will not be fencing on the side of the garden that touches the house.}
    
    \includegraphics[width=5cm]{img/pt3-garden}
    \Smallsp
    
    
    \frq{Bob wants to construct a box whose base length is 3 times the base width. The material used to build the top and bottom cost \$5 per square foot and the material used to build the sides cost \$4 per square foot. If the box must have a volume of $50$ cubic feet, determine the dimensions that will minimize the cost to build the box. {\color{red} You do not need to simplify your answer.}}
    
    \newpage
    
    \begin{multipartquestion}{For the following questions, use the function $f(x)=\dfrac{1}{4}x^5-\dfrac{5}{3}x^3+1$.}
        \frq{What are the critical values of $f$?}
        \ \newline
        \mcq[4]{Give the interval(s) for which $f$ is increasing and decreasing and give the \textbf{coordinates} of any maximum(s) or minimum(s). If there are none, write ``NONE.''}{
            \task Increasing
            \task Decreasing
            \task Maximums
            \task Minimums
        }
        \ \newline
        \ \newline
        \frq{What are the hypercritical values of $f$?}
        \ \newline
        \mcq[3]{Give the interval(s) for which $f$ is concave up and concave down and give the \textbf{coordinates} of any inflection points. If there are none, write ``NONE.''}{
            \task Concave up
            \task Concave down
            \task Inflection points
        }
        \ \newline
        \begin{multicols}{2}
            \frq{What is $\lim\limits_{x\to \infty}f(x)$?}
            \frq{What is $\lim\limits_{x\to -\infty}f(x)$?}
        \end{multicols}
        \ \newline
        \frq{Use your answers to parts (a)-(f) to sketch a precise graph of $f(x)$. Start with the set of axes below and \textbf{label tick marks on both axes \underline{AND} label the coordinates of any transition points}.}
    \end{multipartquestion}

    \begin{center}
        \begin{tikzpicture}
            \draw (8,0) -- (-8,0);
            \draw (0,-4) -- (0,4);
        \end{tikzpicture}
    \end{center}
    
    \newpage
    
    \activity{Section 2}{Additional Practice}{These problems \underline{will not} look like what you will see on the exam, but help to reinforce the skills that you \underline{will} need. These are all problems from my old practice tests, and are generally much harder than you will see on your exam.}{try these problems second}
    
    % Approximations and Linearizations 
    \hspace{-5mm}\textbf{Approximations and Linearizations}
    \begin{multipartquestion}
        Information about $f(x)$ and $f'(x)$ is given in the chart below for certain values of $x$.
        
        \begin{center}
            \begin{tabular}{|c|c|c|c|c|}
                \hline
                $x$ & $-2$ & $-1$ & $0$ & $1$\\
                \hline
                $f(x)$ & 3 & 2 & 0 & -1 \\
                \hline
                $f'(x)$ & $-1/8$ & $-1/3$ & $-1$ & $0$\\
                \hline
            \end{tabular}
        \end{center}
    
        \frq{Determine the linearization of $f(x)$ at $x=-1$.}
        \Smallsp
        \frq{Use your answer from part (a) to estimate $f\left(-\dfrac32\right)$.}
    \end{multipartquestion}
    \newpage
    \frq{Approximate the difference $3.01^2-3^2$ using a linear approximation.}
    \largesp

    % Derivative Properties
    \hspace{-5mm}\textbf{Derivative Properties}
    % FIN-F17 2
    \mcq{Suppose $f'(a)=0$. Mark each of the following statements as true (T) (meaning ``\textit{always} true'') or false (F) (meaning ``\textit{sometimes} false''). You do \textit{not} need to justify your answer.}{
        \task $f$ is continuous at $x=a$.
        \task The line tangent to $y=f(x)$ at $x=a$ is horizontal.
        \task $f$ has a local max or local min at $x=a$.
        \task $\lim\limits_{x\to a}\dfrac{f(x)-f(a)}{x-a}=0$
        \task $f$ has an inflection point at $x=a$.
        \task $f'$ is continuous at $x=a$.
        \task $x=a$ is a critical value of $f$.
        \task The line tangent to $y=f'(x)$ at $x=a$ is horizontal.
        \task $\lim\limits_{h\to 0}\dfrac{f(x+h)-f(x)}{h}=0$
         
    }
\newpage
    % FIN-F17 8 (modified)
    \mcq{Let $f(x)=x^3+4x^2$}{
        \task Find the intervals on which $f$ is increasing/decreasing.
        \smallsp
        \task Find and classify all local extrema of $f$.
        \smallsp
        \task Find the intervals on which $f$ is concave up/down.
        \smallsp
        \task Find all inflection points of $f$.
        \smallsp
    }
    \newpage
    % FIN-S19 11
    \begin{multipartquestion}
        The function $f$ is continuous for all values of $x$. Information about the sign of $f'$ and $f''$ is organized in the table below.
        
        \begin{center}
            \begin{tabular}{|c|c|c|c|c|}
                \hline
                &$x<1$&$1<x<2$&$2<x<3$&$x>3$\\
                \hline
                Sign of $f'$&$-$&$-$&$+$&$-$\\
                \hline
                Sign of $f''$&$+$&$-$&$-$&$+$\\
                \hline
            \end{tabular}
        \end{center}
        
        Mark each of the following statements as true (T) or false (F). You do \textit{not} need to justify your answer.
        \frq{$f$ has a local minimum at $x=2$}
        \frq{$f$ is decreasing and concave down at $x=4$}
        \frq{$f$ has an inflection point at $x=1$}
        \frq{$f'$ is decreasing at $x=2.5$}
        \frq{$f'$ has a local extremum at $x=1$}
    \end{multipartquestion}
    \newpage
    % FIN-F18 4
    \begin{multipartquestion}
        Researchers are studying the effectiveness of a new antibiotic. Let $P(t)$ be the number of bacteria present $t$ hours after the antibiotic is administered. Data collected by the researchers indicated that the over the first $12$ hours, the number of bacteria decreased while its rate of change increased, and then afterwards the number of bacteria continued to decrease, but its rate of change decreased.
        \mcq{Which first derivative statements best describe this scenario?}{
            \task $P'(t) > 0$ for $0 < t < 12$ and $P'(t) > 0$ for $t > 12$
            \task $P'(t) > 0$ for $0 < t < 12$ and $P'(t) < 0$ for $t > 12$
            \task $P'(t) < 0$ for $0 < t < 12$ and $P'(t) > 0$ for $t > 12$
            \task $P'(t) < 0$ for $0 < t < 12$ and $P'(t) < 0$ for $t > 12$
            \task None of the above
        }
        \mcq{Which second derivative statements best describe this scenario?}{
            \task $P''(t) > 0$ for $0 < t < 12$ and $P''(t) > 0$ for $t > 12$
            \task $P''(t) > 0$ for $0 < t < 12$ and $P''(t) < 0$ for $t > 12$
            \task $P''(t) < 0$ for $0 < t < 12$ and $P''(t) > 0$ for $t > 12$
            \task $P''(t) < 0$ for $0 < t < 12$ and $P''(t) < 0$ for $t > 12$
            \task None of the above
        }
        \mcq{At $t=12$, the graph of $y=P(t)$ has}{
            \task a local maximum
            \task a local minimum
            \task a critical number but no local extremum
            \task an inflection point
            \task None of the above
        }
        \frq{Assuming that there are a positive number of bacteria, $P_0$, present at time $t = 0$ and that at time $t = 24$, all bacteria are eliminated, sketch a possible graph of $P$ corresponding to all the criteria described above.}
    \end{multipartquestion}
    \newpage
    % REC-10 1
    \begin{multipartquestion}
        For the graph below, determine the following. Assume the domain of $f(x)$ is all real numbers.
        \begin{center}
            \includegraphics[width=8cm]{ShapeOfGraph}
        \end{center}
        \frq{$f(x)$ is increasing on:}
        \tinysp
        \frq{$f(x)$ is decreasing on:}
        \tinysp
        \frq{$x$-coordinates of local minima}
        \tinysp
        \frq{$x$-coordinates of local maxima}
        \tinysp
        \frq{$f(x)$ is concave up on:}
        \tinysp
        \frq{$f(x)$ is concave down on:}
        \tinysp
        \frq{$x$-coordinates of points of inflection:}
        \tinysp
    \end{multipartquestion}
\newpage
    % REC-10 3
    \begin{multipartquestion}
        \frq{Consider a function $f(x)$ with continuous second derivative. If $f'(2)=0$ and $f''(2)=0$, can we tell if $f$ has a local extrema at $x=2$? If so, is it a maximum or a minimum and why?}
        \Tinysp
        \frq{Consider a function $f(x)$ with continuous second derivative. If $f'(7)=0$ and $f''(7)<0$, can we tell if $f$ has a local extrema at $x=7$? If so, is it a maximum or a minimum and why?}
        \Tinysp
        \frq{Consider a function $f(x)$ with continuous second derivative. If $f'(-3)=0$ and $f''(-3)>0$, can we tell if $f$ has a local extrema at $x=-3$? If so, is it a maximum or a minimum and why?}
        \Tinysp
    \end{multipartquestion}
    % UGA-F18-3 1
    \frq{If applicable, use the Extreme Value Theorem to determine the absolute maximum and minimum values of the function on the given interval. Be sure to justify your use of the Extreme Value Theorem.}
    $$f(x)=\cos(x)-\sin(x)\text{    on the interval }[0,\pi]$$
\newpage
    
    % FIN-F18 7
    \mcq{Suppose $f'(x)=x(x-1)^2$.}{
        \task Find the $x$-coordinate for all local extrema of $f$. Classify each as where a local max or local min occurs.
        \Normalsp
        \task Find the $x$-coordinate for all inflection points of $f$.
        \Normalsp
    }
    % Mean Value Theorem
    \hspace{-5mm}\textbf{Mean Value Theorem}
    % FIN-F17 11
    \frq{Consider $f(x)=x-\dfrac1x$ on $1\leq x\leq 2$. Find a value of $c$ guaranteed by the Mean Value Theorem.}
    \newpage
    
    % Optimization
    \hspace{-5mm}\textbf{Optimization}
    \frq{A rectangle is inscribed in the first quadrant region bounded by the $x$-axis, the $y$-axis, and the parabola $y=16-x^2$. That is, the base of the rectangle is along the $x$-axis, its lower left corner is at the origin, and its upper right corner is on the parabola $y=16-x^2$. Find the length and width of the rectangle of greatest perimeter.}
    \hugesp
%    \frq{We want to construct a box whose base length is 3 times the base width. The material used to build the top and bottom cost \$5 per square foot and the material used to build the sides cost \$4 per square foot. If the box must have a volume of $50$ cubic feet, determine the dimensions that will minimize the cost to build the box.}
%    \hugesp
%    \frq{Find the points on the graph $y=4-2x^2$ that are closest to $(0,-4)$. Hint: remember that the distance between the points $(x,y)$ and $(a,b)$ is $d=\sqrt{(x-a)^2+(y-b)^2}$. \textbf{Also remember that minimizing $d$ is the same as minimizing $d^2$.}}
%    \hugesp
    \frq{Find the maximum volume of a cylinder whose surface area is $300\pi$ square meters.}
    \newpage
    
    % Miscellaneous
    \hspace{-5mm}\textbf{Miscellaneous}
    
    \begin{multipartquestion}
        You begin to heat a pot of water on the stove. At time $t$ (in minutes), the temperature $T$ (in $^{\circ}$F) of the water is recorded below. For $0 \leq t \leq 8$, $T$ is a differentiable function of $t$.
        \begin{center}
            \begin{tabular}{|l|r|r|r|r|r|}
                \hline
                $t$ (min) & 0 & 1 & 3 & 4 & 8 \\
                \hline
                $T$ ($^{\circ}$F) & 100 & 110 & 140 & 160 & 180\\
                \hline
            \end{tabular}
        \end{center}
        \frq{Find the average rate of change of the temperature of the water over $0\leq t\leq8$. Include units.}
        \Smallsp
        \frq{Was there some time $t$ between $t = 0$ and $t = 8$ when the instantaneous rate of change of the temperature of the water was $10$$^{\circ}$F/min? Explain why or why not.}
        \Normalsp
        \frq{Assuming that $T'(4)=10$, estimate $T(6)$ using a method of your choice.}
        \newpage
    \end{multipartquestion}
    \mcq{Let $f(x)=\dfrac{1}{16}x^2-\dfrac{x+1}{x}$.}{
        \task Find the absolute maximum and minimum values of $f$ on the interval $[-3,-1]$.
        \hugesp
        \task Find all infection points of $f(x)$ (if any exist).
    }
    
\end{document}