\documentclass{siproblemset}

\usepackage{multicol}
\usepackage{amsmath}

% SI Session Information
\course{MTH 1321}
\sessionnum{3}
\sessiondate{1/27/22}

\warmup{Concept Review}
\topic{Limit Laws}
\topic{Determining Continuity}
\cooldown{Challenge Problems}

% Worksheet Information
\title{Limit Laws}
\sections{Sections 2.3}
\withnamespace

\begin{document}
    \maketitle
    
    \activity{Warmup}{Concept Review}{Work \textbf{alone} then share your answers with a \textbf{partner}. Try not to use your notes.}{15 minutes}
    
    \frq{What are the basic limit laws?}
    \begin{center}
        \begin{tabular}{ |c|c|c| } 
            \hline
            \multicolumn{3}{|c|}{If $\lim\limits_{x\to c}f(x)=L$ and $\lim\limits_{x\to c}g(x)=M$ exist, then:}\\[4ex]
            \hline
            \textbf{Law Name} & \textbf{Law Definition} & \textbf{Constraints (if applicable)} \\ 
            \hline
            &&\\
            Constant Law & \hspace{3in} & \\ 
            &&\\
            \hline
            &&\\
            Identity Law & \hspace{3in} & \\ 
            &&\\
            \hline
            &&\\
            Sum Law & \hspace{3in} & \\ 
            &&\\
            \hline
            &&\\
            Difference Law & \hspace{3in} & \\ 
            &&\\
            \hline
            &&\\
            Constant Multiple Law & \hspace{3in} & \\ 
            &&\\
            \hline
            &&\\
            Product Law & \hspace{3in} & \\ 
            &&\\
            \hline
            &&\\
            Quotient Law & \hspace{3in} & \\ 
            &&\\
            \hline
            &&\\
            Power-Root Law & \hspace{3in} & \\ 
            &&\\
            \hline
        \end{tabular}
    \end{center}
    \pagebreak
    
    \activity{Activity 1}{Solving Limits Using Limit Laws}{Work in a \textbf{group of 2 or 3 people with the same color worksheet as you} to answer these questions. Try not to use your notes.}{15 minutes}
    \mcq{Evaluate the following limits by using the Basic Limit Laws. \textit{You do not need to simplify.}}{
        \task $\lim\limits_{x\to25}\dfrac{3\sqrt{x}-\frac15x}{(x-20)^2}$
        \hugesp
        \task $\lim\limits_{x\to 4}\dfrac{x^2+1}{\left(x^3+2\right)\left(x^4+1\right)}$
    }
    \pagebreak
    
    \activity{Activity 2}{Solving Limits Using Limit Laws}{Work in a \textbf{group of 2 or 3 people with a different color worksheet as you} to answer these questions. Try not to use your notes.}{15 minutes}
    \mcq{Evaluate the following limits by using the Basic Limit Laws. \textit{You do not need to simplify.} $$\lim\limits_{x\to2}g(x)=3~~~\text{and}~~~\lim\limits_{x\to2}h(x)=-2$$}{
        \task $\lim\limits_{x\to2}\dfrac{g(x)+5x}{h(x)^2}$
        \normalsp
        \task $\lim\limits_{x\to2}\frac12h(x)\sqrt{g(x)}$
        \normalsp
        \task $\lim\limits_{x\to2}\dfrac{g(x)^2}{g(x)-1}$
        \normalsp
        \task $\lim\limits_{x\to2}\dfrac{g(x)^{3/2}}{2h(x)}$
    }

    ~
    \normalsp
    
    \activity{Cooldown}{Challenge Problems}{Test your knowledge of previous topics and upcoming topics by answering these questions \textbf{alone}. Try not to use your notes.}{15 minutes}
    
    \frq{State two functions $f(x)$ and $g(x)$ such that both $\lim\limits_{x\to1}f(x)$ and $\lim\limits_{x\to1}g(x)$ do not exist, but $\lim\limits_{x\to1}f(x)-2g(x)$ does exist.}
    \Normalsp
    \frq{State two functions $f(x)$ and $g(x)$ such that both $\lim\limits_{x\to0}f(x)$ and $\lim\limits_{x\to1}g(x)$ do not exist, but $\lim\limits_{x\to1}f(x)+g(x)$ does exist.}    
\end{document}