\documentclass{siproblemset}

% SI Session Information
\course{MTH 1321}       % the course of your SI
\sessionnum{idk}         % (optional) specify the session number
\sessiondate{11/16/20}   % the date of the session

\warmup{Concept Review}
\topic{Summation Notation}
\topic{Approximating Area Under a Curve}
\topic{Approximating Area With Unequal Partitions}
\cooldown{Review}

% Worksheet Information
\title{Approximating Area \linebreak Under a Curve}
\sections{Section 5.1}
\withnamespace

\begin{document}
    \maketitle
    
    \activity{Warmup}{Concept Review}{Work \textbf{alone} to answer these questions. Try not to use your notes.}{10 minutes}
    
    \mcq{Draw pictures of what each of the following would look like on a graph:}{
        \task left-endpoint equal-width approximation ($L_n$)
        \task midpoint equal-width approximation ($M_n$)
        \task right-endpoint equal-width approximation ($R_n$)
    }
    \Normalsp
    \frq{The area under a velocity vs. time graph gives \underline{\hspace{2in}}.}
    \tinysp
    \frq{The area under an acceleration vs. time graph gives \underline{\hspace{2in}}.}
    \tinysp
        \newpage
    \activity{Activity 1}{Summation Notation}{Make a \textbf{group of two or three, all with the same colored worksheets}, to work together to answer these questions. Try not to use your notes. \textbf{DO NOT use a calculator}.}{15 minutes}
    \frq{Compute $\sum_{j=3}^{6}\sin\left(j\frac\pi2\right)$.}
    \Normalsp
    \frq{Compute $\sum_{k=2}^{4}k^3$.}
    \Normalsp
    \frq{Express in summation notation: $\dfrac{1}{2\cdot3}+\dfrac{2}{3\cdot4}+\dfrac{3}{4\cdot5}+\dfrac{4}{5\cdot6}$.}
    \newpage
    \activity{Activity 2}{Approximating Area Under a Curve}{Make a \textbf{group of two or three, all with the same colored worksheets}, to work together to do your assigned parts. Then, try to do the other parts. Try not to use your notes. \textbf{DO NOT use a calculator}.}{30 minutes}
    
    \frq{Approximate the area under the function $f(x)=20-3x$ on $[2,6]$ using 2 and 4 rectangles using left-endpoint or right-endpoint approximations. Then, compute the actual area $A$ of the region using geometry.}
    \Hugesp
    \smallsp
    
    \frq{If you have extra time, compute $\abs{A-L_4}$ or $\abs{A-R_4}$ (depending on which sum you computed above). These values are called the \textit{error} and are a measure of how close your approximation was. For those curious, the \textit{percent error} can be computed using the formula $\abs{\dfrac{L_4-A}{A}}\times100\%$.}
    \newpage

    \activity{Activity 3}{Approximations with Unequal Partitions}{Make a \textbf{group of two or three, all with different colored worksheets}, to answer your assigned question. Then, try to answer the other questions. Try not to use your notes. \textbf{DO NOT use a calculator}.}{30 minutes}
    
    \frq{Compute $R_6$, $L_6$, and $M_3$ to estimate the distance traveled over $[0,3]$ if the velocity at half-second intervals is as follows:}
    \begin{center}
        \begin{tabular}{|l|c c c c c c c|}
            \hline
            $t$ (s) & 0 & 0.5 & 1 & 1.5 & 2 & 2.5 & 3\\
            \hline
            $v$ (m/s) & 0 & 12 & 18 & 25 & 20 & 14 & 20\\
            \hline
        \end{tabular}
    \end{center}
\Hugesp
    
    \activity{Cooldown}{Review}{Work \textbf{alone} to answer these questions. Try not to use your notes.}{15 minutes}
        
    \frq{Which approximation strategy is best: $L_n$, $R_n$, or $M_n$?}
    \tinysp
    \frq{How many rectangles are needed to get the actual area?}
    \tinysp
    \frq{Compute $\sum_{i=0}^{5}4$.}
\end{document}