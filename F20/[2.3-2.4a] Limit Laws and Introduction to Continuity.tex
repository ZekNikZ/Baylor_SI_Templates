\documentclass{siproblemset}

\usepackage{multicol}
\usepackage{amsmath}

% SI Session Information
\course{MTH 1321}
\sessionnum{3}
\sessiondate{9/2/20}

\warmup{Concept Review}
\topic{Limit Laws}
\topic{Determining Continuity}
\cooldown{Classifying Discontinuities}

% Worksheet Information
\title{Limit Laws and\linebreak Introduction to Continuity}
\sections{Sections 2.3-2.4a}
\withnamespace

\begin{document}
    \maketitle
    
    \activity{Warmup}{Concept Review}{Work \textbf{alone} then share your answers with a \textbf{partner}. Try not to use your notes.}{15 minutes}
    
    \frq{What are the basic limit laws?}
    \begin{center}
        \begin{tabular}{ |c|c|c| } 
            \hline
            \multicolumn{3}{|c|}{If $\lim\limits_{x\to c}f(x)=L$ and $\lim\limits_{x\to c}g(x)=M$ exist, then:}\\[4ex]
            \hline
            \textbf{Law Name} & \textbf{Law Definition} & \textbf{Constraints (if applicable)} \\ 
            \hline
            Constant Law & \hspace{3in} & \\ 
            &&\\
            \hline
            Identity Law & \hspace{3in} & \\ 
            &&\\
            \hline
            Sum Law & \hspace{3in} & \\ 
            &&\\
            \hline
            Difference Law & \hspace{3in} & \\ 
            &&\\
            \hline
            Constant Multiple Law & \hspace{3in} & \\ 
            &&\\
            \hline
            Product Law & \hspace{3in} & \\ 
            &&\\
            \hline
            Quotient Law & \hspace{3in} & \\ 
            &&\\
            \hline
            Power-Root Law & \hspace{3in} & \\ 
            &&\\
            \hline
        \end{tabular}
    \end{center}

    \frq{What are the three conditions for a function to be continuous at a point?}
    \normalsp
    
    \frq{What does it mean for a \textit{function} to be continuous everywhere?}
    \tinyspace

    \pagebreak
    
    \activity{Activity 1}{Solving Limits Using Limit Laws}{Work in a \textbf{group of 2 or 3 people with the same color worksheet as you} to answer these questions. Try not to use your notes.}{40 minutes}
    \mcq{Evaluate the following limits by using the Basic Limit Laws. \textit{Indicate what law you are using at each part.}}{
        \task $\lim\limits_{x\to25}\dfrac{3\sqrt{x}-\frac15x}{(x-20)^2}$
        \hugesp
        \task $\lim\limits_{x\to 4}\dfrac{x^2+1}{\left(x^3+2\right)\left(x^4+1\right)}$
    }
    \pagebreak
    \mcq{Evaluate the following limits by using the Basic Limit Laws. \textit{Indicate what law you are using at each part.}$$\lim\limits_{x\to2}g(x)=3~~~\text{and}~~~\lim\limits_{x\to2}h(x)=-2$$}{
        \task $\lim\limits_{x\to2}\dfrac{g(x)+5x}{h(x)^2}$
        \normalsp
        \task $\lim\limits_{x\to2}\frac12h(x)\sqrt{g(x)}$
        \normalsp
        \task $\lim\limits_{x\to2}\dfrac{g(x)^2}{g(x)-1}$
        \normalsp
        \task $\lim\limits_{x\to2}\dfrac{g(x)^{3/2}}{2h(x)}$
    }
    \pagebreak
    
    \activity{Activity 2}{Determining Continuity}{Work in a \textbf{group of 2 or 3 people with the same color worksheet as you} to answer these questions. Try not to use your notes.}{20 minutes}
    \mcq{Determine if the following functions are continuous at the indicated values of $a$. If the function is discontinuous at $x=a$, classify the type of discontinuity.}{
        \task $f(x)=2\sin x+3\cos x ~;~ a=\dfrac\pi2$
        \normalsp
        \task $g(x)=\dfrac{x^2-9}{x-3} ~;~ a=3$
        \normalsp
        \task $h(x)=\tan x ~;~ a=-\dfrac\pi2$
        \normalsp
        \task $k(x)=\begin{cases}x^2-1 & x\leq 2\\\dfrac1x & x > 2\end{cases} ~;~ a=2$
        \Normalsp
    }
    
    \activity{Cooldown}{Classifying Discontinuities}{Test your knowledge of previous topics and upcoming topics by answering these questions \textbf{alone}. Try not to use your notes.}{15 minutes}
    
    \frq{What are $\lim\limits_{x\to c^\pm}f(x)$ and $f(c)$ for each type of discontinuity?}
    \pagebreak
\end{document}